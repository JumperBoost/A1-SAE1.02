\documentclass{report}

\usepackage[utf8]{inputenc}
\usepackage[a4paper,includeheadfoot,margin=2.54cm, top=0.5in, bottom=0.5in]{geometry}
\usepackage{xcolor, hyperref, lipsum, graphicx}
\usepackage{amsmath, amsthm, amsfonts, textgreek}
\usepackage{listings}
\usepackage{color}
\usepackage{textcomp}
\usepackage{xinttools}
\usepackage{color}
\usepackage{listings}
\usepackage{caption, float}
\usepackage[most]{tcolorbox}
\usepackage{physics}

\captionsetup[lstlisting]{skip=7pt}
\renewcommand\lstlistingname{Code}
\renewcommand\lstlistlistingname{Code}

\definecolor{darkred}{rgb}{0.6,0.0,0.0}
\definecolor{darkgreen}{rgb}{0,0.50,0}
\definecolor{lightblue}{rgb}{0.0,0.42,0.91}
\definecolor{orange}{rgb}{0.99,0.48,0.13}
\definecolor{grass}{rgb}{0.18,0.80,0.18}
\definecolor{pink}{rgb}{0.97,0.15,0.45}

\definecolor{mygreen}{rgb}{0,0.6,0}
\definecolor{mygray}{rgb}{0.5,0.5,0.5}
\definecolor{mymauve}{rgb}{0.58,0,0.82}


\lstset{language=java,
aboveskip=1em,
breaklines=true,
abovecaptionskip=-6pt,
frame=single,
numbers=left,
numbersep=15pt,
numberstyle=\tiny,
backgroundcolor=\color{white},   % choose the background color
basicstyle=\footnotesize,        % size of fonts used for the code
breaklines=true,                 % automatic line breaking only at whitespace
captionpos=b,                    % sets the caption-position to bottom
commentstyle=\color{mygreen},    % comment style
escapeinside={\%*}{*)},          % if you want to add LaTeX within your code
keywordstyle=\color{blue},       % keyword style
stringstyle=\color{mymauve},     % string literal style
}


\usepackage{fancyvrb}
\usepackage{fancyhdr, lastpage}
\pagestyle{fancyplain}% <- pagestyle fancyplain
\renewcommand\plainheadrulewidth{.4pt}% headrule on plain pages
\lhead{Marche aléatoire}
\rhead{Compte rendu - 2022}
\cfoot{Page \thepage\ of \pageref{LastPage}}
\usepackage{setspace}
\onehalfspacing


\usepackage{titlesec}


\titlespacing*{\section}
{0pt}{5.5ex plus 1ex minus .2ex}{4.3ex plus .2ex}
\titlespacing*{\subsection}
{0pt}{5.5ex plus 1ex minus .2ex}{4.3ex plus .2ex}

\title{\textbf{SAE 1.02 - E3CETE}}
\author{J. Renaud - M. Franceus-Cointrel}
\date{Pour le 14 janvier 2024}


\addtolength{\jot}{1em}

\hypersetup{
  pdftitle={SAE 1.02 : E3Cete - 2023/2024},
  pdfauthor={Renaud Julien},
  pdfsubject={Dev. Init.}
}

\renewcommand{\contentsname}{Table des matières}
\renewcommand{\appendixname}{Annexe}
\renewcommand{\chaptername}{Chapitre}


\begin{document}

\maketitle
\tableofcontents

\chapter*{Introduction}

\qquad Dans cette SAE nous étudions...




\chapter{Analyse et comparaison des 3 méthodes de tris}

\qquad 

\section{Les fonctions de tris et comptage du nombre d'opérations approximatif}

\subsection{Tri par sélection}

\begin{lstlisting}[language=java, caption={\it Focntion marche aléatoire}, label=code1]
%insert code here
\end{lstlisting}

\subsection{Tri par bulles}

\subsection{Tri par insertion}

\pagebreak



\subsection{Initialisation des variables}

L'objectif de cette expérience est d'évaluer la performance de chacun des tris. Pour cela nous devons réaliser chacun des tris en variant le nombre de cartes contenus dans le Paquet.
\bigskip



Les variables fixés : \\
\begin{itemize}

	\item int $nbRepetTest = 1000$
	\item int $cardCouleurs = 1$
	\item int $cardFigures = 1$
	\item int $cardTextures = 1$
	
	\item int $cardRepetFigures \in [\, 10-500 ]\, $
	
	
	
\end{itemize}

Protocole expérimental :
\begin{itemize}
	\item Réaliser les 3 tris sur 1000 paquets ayant les mêmes caractéristiques mais mélangés différemment, pour un même nombre de carte $N$.
	\item Récupérer les nombre d'opérations moyens et le temps d'éxecution moyen des 
\end{itemize}

\boldmath
\begin{equation}
  x_{N+1}=x_{N}\pm \delta
\end{equation}
\unboldmath



\subsection{Analyse graphique}



\chapter{Class \it Table}

























\appendix
\chapter{Modules utilisés}

\begin{lstlisting}[caption={\it Modules utilisés en Python 3.9.2}, label=annexe]
  import math
    from math import pi
  import random as rnd
  import statistics as stats
  import matplotlib
  import matplotlib.pyplot as plt
  import matplotlib.cm as cm
    from matplotlib.patches import Ellipse
    from mpl_toolkits import mplot3d
  import numpy as np
  import scipy as scp
  import scipy.stats as st
    from scipy.stats import multivariate_normal

\end{lstlisting}

\chapter{Sources}

\begin{itemize}
  \item \url{https://moodle.umontpellier.fr/course/view.php?id=25363}
  \item \url{https://femto-physique.fr/physique_statistique/diffusion-moleculaire.php}
  \item \url{https://stringfixer.com/fr/Random_walk}
  \item \url{https://en.wikipedia.org/wiki/Mass_diffusivity}
  \item \url{https://fr.wikipedia.org/wiki/Mouvement_brownien}
  \item \url{https://fr.wikipedia.org/wiki/Lois_de_Fick}
  \item \url{https://fr.wikipedia.org/wiki/Cha%C3%AEne_id%C3%A9ale}
  \item \url{https://fr.wikipedia.org/wiki/Loi_multinomiale}
  \item \url{https://en.wikipedia.org/wiki/Multivariate_normal_distribution}
  \item \url{https://www.youtube.com/channel/UCmpptkXu8iIFe6kfDK5o7VQ}
  \item \url{https://www.caam.rice.edu/~heinken/latex/symbols.pdf}
  \item \url{https://matplotlib.org/stable/index.html}
\end{itemize}

\begin{figure}
  \centering

  \caption{\it Langages et éditeurs utilisés}
\end{figure}





\end{document}









